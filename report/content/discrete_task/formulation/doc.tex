\subsection{Формализация задачи}

Реальные системы не могут передавать данные о своём положении непрерывно.
В данном разделе мы будем считать, что данные о состоянии передаются с некоторым известным детерминированным интервалом времени $\varepsilon$.

В таком случае кажутся естественным провести редукцию системы \eqref{eq:first_task} к дискретному виду.
Обозначим за $\{t^k\,:\,t^{k+1} > t^k\}_{k=1}^N$ времена поступления наблюдений, а за $x^k = x(t^k)$ --- соответсвующие состояния системы \eqref{eq:first_task}.
Будем строить кусочно-постоянное управление вида
$$
        u(t) \equiv u^{k},
        \quad
        t^{k} + h \leqslant t \leqslant t^{k+1} + h.
$$

\begin{figure}[h]
        \centering
        

\tikzset{every picture/.style={line width=0.75pt}} %set default line width to 0.75pt        

\begin{tikzpicture}[x=0.75pt,y=0.75pt,yscale=-1,xscale=1]
%uncomment if require: \path (0,300); %set diagram left start at 0, and has height of 300

%Straight Lines [id:da4879951122895062] 
\draw    (10,90) -- (368,90) ;
\draw [shift={(370,90)}, rotate = 180] [color={rgb, 255:red, 0; green, 0; blue, 0 }  ][line width=0.75]    (10.93,-3.29) .. controls (6.95,-1.4) and (3.31,-0.3) .. (0,0) .. controls (3.31,0.3) and (6.95,1.4) .. (10.93,3.29)   ;
%Straight Lines [id:da6973519196452055] 
\draw    (40,50) -- (40,88) ;
\draw [shift={(40,90)}, rotate = 270] [color={rgb, 255:red, 0; green, 0; blue, 0 }  ][line width=0.75]    (10.93,-3.29) .. controls (6.95,-1.4) and (3.31,-0.3) .. (0,0) .. controls (3.31,0.3) and (6.95,1.4) .. (10.93,3.29)   ;
%Straight Lines [id:da37258372717815214] 
\draw    (120,50) -- (120,88) ;
\draw [shift={(120,90)}, rotate = 270] [color={rgb, 255:red, 0; green, 0; blue, 0 }  ][line width=0.75]    (10.93,-3.29) .. controls (6.95,-1.4) and (3.31,-0.3) .. (0,0) .. controls (3.31,0.3) and (6.95,1.4) .. (10.93,3.29)   ;
%Straight Lines [id:da6922585100312479] 
\draw    (200,50) -- (200,88) ;
\draw [shift={(200,90)}, rotate = 270] [color={rgb, 255:red, 0; green, 0; blue, 0 }  ][line width=0.75]    (10.93,-3.29) .. controls (6.95,-1.4) and (3.31,-0.3) .. (0,0) .. controls (3.31,0.3) and (6.95,1.4) .. (10.93,3.29)   ;
%Straight Lines [id:da8649476764757188] 
\draw    (43,120) -- (117,120) ;
\draw [shift={(120,120)}, rotate = 180] [fill={rgb, 255:red, 0; green, 0; blue, 0 }  ][line width=0.08]  [draw opacity=0] (8.93,-4.29) -- (0,0) -- (8.93,4.29) -- cycle    ;
\draw [shift={(40,120)}, rotate = 0] [fill={rgb, 255:red, 0; green, 0; blue, 0 }  ][line width=0.08]  [draw opacity=0] (8.93,-4.29) -- (0,0) -- (8.93,4.29) -- cycle    ;
%Straight Lines [id:da07027249230414545] 
\draw    (123,120) -- (197,120) ;
\draw [shift={(200,120)}, rotate = 180] [fill={rgb, 255:red, 0; green, 0; blue, 0 }  ][line width=0.08]  [draw opacity=0] (8.93,-4.29) -- (0,0) -- (8.93,4.29) -- cycle    ;
\draw [shift={(120,120)}, rotate = 0] [fill={rgb, 255:red, 0; green, 0; blue, 0 }  ][line width=0.08]  [draw opacity=0] (8.93,-4.29) -- (0,0) -- (8.93,4.29) -- cycle    ;
%Straight Lines [id:da49306868755986777] 
\draw  [dash pattern={on 4.5pt off 4.5pt}]  (40,90) -- (40,130) ;
%Straight Lines [id:da4941954626905687] 
\draw  [dash pattern={on 4.5pt off 4.5pt}]  (120,90) -- (120,130) ;
%Straight Lines [id:da435515110639233] 
\draw  [dash pattern={on 4.5pt off 4.5pt}]  (200,90) -- (200,130) ;
%Straight Lines [id:da4247825385679421] 
\draw    (280,50) -- (280,88) ;
\draw [shift={(280,90)}, rotate = 270] [color={rgb, 255:red, 0; green, 0; blue, 0 }  ][line width=0.75]    (10.93,-3.29) .. controls (6.95,-1.4) and (3.31,-0.3) .. (0,0) .. controls (3.31,0.3) and (6.95,1.4) .. (10.93,3.29)   ;
%Straight Lines [id:da6033872749323609] 
\draw  [dash pattern={on 4.5pt off 4.5pt}]  (280,90) -- (280,130) ;
%Straight Lines [id:da04263965775882406] 
\draw    (203,120) -- (277,120) ;
\draw [shift={(280,120)}, rotate = 180] [fill={rgb, 255:red, 0; green, 0; blue, 0 }  ][line width=0.08]  [draw opacity=0] (8.93,-4.29) -- (0,0) -- (8.93,4.29) -- cycle    ;
\draw [shift={(200,120)}, rotate = 0] [fill={rgb, 255:red, 0; green, 0; blue, 0 }  ][line width=0.08]  [draw opacity=0] (8.93,-4.29) -- (0,0) -- (8.93,4.29) -- cycle    ;
%Straight Lines [id:da5149395390087544] 
\draw [color={rgb, 255:red, 255; green, 255; blue, 255 }  ,draw opacity=1 ]   (0,0) -- (390,0) ;

% Text Node
\draw (377,82.4) node [anchor=north west][inner sep=0.75pt]    {$t$};
% Text Node
\draw (31,30.4) node [anchor=north west][inner sep=0.75pt]    {$x^{k}$};
% Text Node
\draw (111,30.4) node [anchor=north west][inner sep=0.75pt]    {$x^{k+1}$};
% Text Node
\draw (191,32.4) node [anchor=north west][inner sep=0.75pt]    {$x^{k+2}$};
% Text Node
\draw (11,52.4) node [anchor=north west][inner sep=0.75pt]    {$\dotsc $};
% Text Node
\draw (311,52.4) node [anchor=north west][inner sep=0.75pt]    {$\dotsc $};
% Text Node
\draw (77,102.4) node [anchor=north west][inner sep=0.75pt]    {$\varepsilon $};
% Text Node
\draw (157,102.4) node [anchor=north west][inner sep=0.75pt]    {$\varepsilon $};
% Text Node
\draw (271,32.4) node [anchor=north west][inner sep=0.75pt]    {$x^{k+3}$};
% Text Node
\draw (237,102.4) node [anchor=north west][inner sep=0.75pt]    {$\varepsilon $};


\end{tikzpicture}

        \caption{Иллюстрация перехода к дискретной системе. Здесь показано в какие моменты времени наблюдается состояние системы и на каких промежутках действуют постоянные управления.}
\end{figure}

Применим на промежутке $t^k \leqslant t \leqslant t^{k+1}$ формулу Коши:
\begin{multline*}
x^{k+1} = X(t^{k+1},\,t^k)x^k
+
\int\limits_{t^k}^{t^{k} + \hat h} X(t^{k} + \hat h,\,s)B(s)\,ds\cdot u^{k-m-1}
+\\+
\int\limits_{t^{k} +\hat h}^{t^{k+1}} X(t^{k+1},\,s)B(s)\,ds\cdot u^{k-m},
\end{multline*}
где $m = \left\lfloor\frac{h}{\varepsilon}\right\rfloor$, $\hat h = h - m\varepsilon$.
Таким образом, мы получили дискретную систему с запаздыванием по управлению
\begin{equation}\label{eq:first-discr}
        x^{k+1} = \Phi^k x_k + \Gamma_1^k u^{k-m} + \Gamma_2^k u^{k-m-1}.
\end{equation}

Упростим систему \eqref{eq:first-discr}. На момент времени $t^k$ помимо состояния системы $x^{k}$ нам известны все переданные на тот момент управления $u^0,\,\ldots,\,u^{k-1}$. Это значит, что мы можем рассчитать $x^{k+m}$, последовательно $m$ раз применив формулу \eqref{eq:first-discr} к состоянию $x^k$. Таким образом, не ограничивая общности, можем считать, что величина запаздывания $h$ строго меньше интервала между наблюдениями $\varepsilon$, и система преобразуется к виду
\begin{equation}\label{eq:main-discr}
        x^{k+1} = \Phi^kx^k + \Gamma_1^ku^k + \Gamma_2^ku^{k-1}.
\end{equation}

Для системы \eqref{eq:main-discr} поставим задачу минимизации квадратичного функционала
\begin{equation}\label{eq:main-discr-1}
        J(u) = \sum\limits_{k = 1}^{N}
        \left[
\langle
x^k,\,M^k x^k
\rangle
+
\langle
u^k,\,N^k u^k
\rangle
        \right]
        +
        \langle
        x^{N+1}
        ,\,
        T x^{N+1}
        \rangle
        \to \min\limits_{u},
\end{equation}
где $M^k = (M^k)\T \geqslant 0$, $N^k = (N^k)\T > 0$, $k = \overline{1,N}$ и  $T = T\T > 0$.