\subsection{Решение задачи}

Прежде всего сделаем замену фазовой переменной, чтобы избавиться от запаздывания по управлению в системе \eqref{eq:main-discr}:
$$
        z^k = \begin{pmatrix}
x^k\\
u^{k-1}
        \end{pmatrix}.
$$
Тогда условие задачи \eqref{eq:main-discr}--\eqref{eq:main-discr-1} можно переписать в следующем виде:
\begin{equation}\label{eq:simple-disc}
        z^{k+1} = \underbrace{\begin{pmatrix}
\Phi^k & \Gamma_2^k\\
O & O
        \end{pmatrix}}_{\tilde \Phi^k}
        z^k + \underbrace{
        \begin{pmatrix}
\Gamma_1^k \\
I
        \end{pmatrix}
        }_{\tilde\Gamma^k} u^k,
\end{equation}
\begin{equation}\label{eq:simple-disc-1}
        J(u) = \sum\limits_{k = 1}^{N}
        \left
        \langle
z^k,\,
\underbrace{
\begin{pmatrix}
M^k & O \\
O & O
\end{pmatrix}}_{\tilde M^k}
z^k
        \right
        \rangle
        +
        \sum\limits_{k = 1}^{N}
        \left
        \langle
u^k,\,
N^k
u^k
        \right
        \rangle
        +
        \left
        \langle
z^{N+1},\,
\underbrace{
\begin{pmatrix}
T & O \\
O & O
\end{pmatrix}}_{\tilde T}
z^{N+1}
        \right
        \rangle.    
\end{equation}

\begin{assertion}
        В формуле \eqref{eq:simple-disc-1} матрицы $\tilde M^k \geqslant 0$, $\tilde T > 0$.
\end{assertion}

\begin{proof}
Докажем для матрицы $\tilde T$, для остальных матриц аналогично. По определению
$$
        \langle z,\, \tilde T z \rangle =
        \langle x,\, Tx \rangle + \langle u,\, Ou\rangle = 
        \langle x,\, Tx \rangle > 0.
$$
\end{proof}

Будем действовать так же, как и в случае с непрерывной системой. Введём функцию цены 
$$
V^l(z) = \min\limits_u\left\{
        \sum\limits_{k = l}^{N}
        \left
        \langle
z^k,\,
{\tilde M^k}
z^k
        \right
        \rangle
        +
        \sum\limits_{k = l}^{N}
        \left
        \langle
u^k,\,
N^k
u^k
        \right
        \rangle
        +
        \left
        \langle
z^{N+1},\,
{\tilde T}
z^{N+1}
        \right
        \rangle
\right\},
\;
z^l = z.
$$
Причём $V^1(z^1) = \min\limits_u J(u)$. Такая функция цены удовлетворяет дискретному уравнению Гамильтона--Якоби--Беллмана, которое позволяет определить $V^k(z)$ рекурсивно
\begin{equation}\label{eq:gyb-discr}
V^{k-1}(z) = \langle z,\,\tilde M^k z\rangle
+
\min\limits_{u^k}
\left\{
\langle
u^k,\,N^k u^k
\rangle
+
V^{k}(\tilde\Phi^k z + \tilde\Gamma^k u^k)
\right\},
\end{equation}
с краевым начальным условием
\begin{equation}\label{eq:gyb-discr-1}
        V^{N+1}(z)
        =
        \langle z,\,\tilde T z\rangle.
\end{equation}

Предположим, что функция цены задаётся квадратичной формой $V^{k}(z) = \langle z,\, P^{k} z\rangle$, где $P^{k} = (P^{k})\T > 0$ и покажем, что $V^{k-1}(z)$ будет иметь ту же форму. Из \eqref{eq:gyb-discr} получаем:
$$
V^{k-1}(z) = \langle z,\, \tilde M^k z \rangle + \min\limits_{u^k}\left\{\underbrace{
\langle u^k,\,N^k u^k \rangle
+
\langle
\tilde\Phi^k z + \tilde\Gamma u^k,\,
P^{k}(\tilde\Phi^k z + \tilde\Gamma^k u^k)
\rangle}_{\Psi(z,\,u^k)}
\right\}.
$$
Воспользуемся необходимым условием минимума, приравняв производную 
$\Psi'_{u^k}(u^*,\,z)$ к нулю:
$$
        2 N^k u^{k*} + 2(\tilde\Gamma^k)\T P^k\tilde\Phi^k z + 2(\tilde\Gamma^k)\T P^k \tilde \Gamma ^ku^{k*} = 0,
$$
\begin{equation}\label{eq:opt-discr}
        u^{k*} = -[N^k + (\tilde\Gamma^k)\T P^{k}\tilde\Gamma^k]^{-1}(\tilde\Gamma^k)\T P^{k} \tilde\Phi^k z.
\end{equation}
Тогда функция цены
\begin{align*}
V^{k-1}(z) &=
\langle
z,\,\tilde M^k z
\rangle
+
\langle
u^*,\,N^ku^*
\rangle
+
\langle
\tilde\Phi^k z + \tilde\Gamma^k u^*
,\,P^{k}(\tilde\Phi^k z + \tilde\Gamma^k u^*) =\\
&=
\langle
z,\,
[\tilde M^k
+
(\tilde\Phi^k)\T P^k \tilde\Phi^k
-\\&\qquad\qquad\;\;-
(\tilde\Phi^k)\T P^k \tilde\Gamma^k (N^k + (\tilde\Gamma^k)\T P^k \tilde\Gamma^k)^{-1} (\tilde\Gamma^k)\T P^k \tilde\Phi^k
]z\rangle = \\
&=
\langle z,\, P^{k-1} z \rangle,
\end{align*}
причем, очевидно, что $P^k = (P^k)\T > 0$.

Подведём итог: оптимальную стратегию можно найти используя формулу~\eqref{eq:opt-discr}, где матрица $P^k$ должна удовлетворять следующим соотношениям:
\begin{equation}\label{eq:disc-rikkati}
        \left\{
        \begin{aligned}
P^{k-1} &= \tilde M^k
+
(\tilde\Phi^k)\T P^k \tilde\Phi^k-
(\tilde\Phi^k)\T P^k \tilde\Gamma^k (N^k + (\tilde\Gamma^k)\T P^k \tilde\Gamma^k)^{-1} (\tilde\Gamma^k)\T P^k \tilde\Phi^k,\\
P^{N+1} &= \tilde T.
        \end{aligned}
        \right.
\end{equation}