\subsection{Численный синтез управления}

Приведем алгоритм для построения управления:
\begin{enumerate}
        \item Переведем систему к дискретному виду, посчитав соответствующие матрицы 
        $\{\Phi^k\}_{k=1}^{N}$, $\{\Gamma_1^k\}_{k=1}^{N}$, $\{\Gamma_2^k\}_{k=1}^{N}$.

        \item Запомним матрицы $\{P^k\}_{k=1}^N$, полученные из соотношений \eqref{eq:disc-rikkati}.

        \item На каждом шаге работы алгоритма будем переводить $x^k \longrightarrow x^{k+m}$ так, чтобы величина запаздывания $h$ стала меньше интервала между наблюдениями $\varepsilon$. Причём, это можно сделать ``прогнав через систему'' положение $x^{k+m-1}$, посчитанное на предыдущей итерации алгоритма, то есть применив формулу
$$
        x^{k+m} = \Phi^{k+m-1}x^{k+m-1} + \Gamma_1^{k+m-1}u^{k-1} + \Gamma_2^{k+m-1}u^{k-2},
$$
        тем самым исключив зависимость времени работы программы от величины задержки $h$.

        \item Найдём управление $u^k$ из выражения \eqref{eq:opt-discr}.
\end{enumerate}
В отличие от алгоритма без запаздывания построенный нами алгоритм предполагает дополнительное хранение предыдущего состояния и предыдущего управления.