\subsection{Формализация задачи}

Рассмотрим линейную динамическую систему с непрерывными ограниченными коэффициентами
\begin{equation}\label{eq:first_task}
        \frac{d}{dt}x(t) = A(t)x(t) + B(t)u(t),
        \quad
        t_0 \leqslant t \leqslant t_1
\end{equation}
с заданным начальным условием
\begin{equation}\label{eq:start_condition}
        x(t_0) = x^0.
\end{equation}

Для задачи Коши~\eqref{eq:first_task}--\eqref{eq:start_condition} поставим задачу поиска измеримого управления, минимизируещего следующий интегрально-квадратичный функционал
\begin{equation}\label{eq:functional}
        J(u)
=
        \int\limits_{t_0}^{t_1}
[
\langle
x,\,M(s)x
\rangle
+
\langle
u,\,N(s)u
\rangle
]\,ds
        +
        \langle
        x(t_1),\,Tx(t_1)
        \rangle
\to
        \min\limits_{u \in U},
\end{equation}
где $M(s) = M\T(s) \geqslant 0$,
$N(s) = N\T(s) > 0$,
$T > 0$,
а $U = U[t_0,\,t_1]$ --- мно\-жес\-т\-во измеримых функций на отрезке $t_0 \leqslant t \leqslant t_1$.

Допустимый класс управления выбран таким образом, чтобы выполнялись достаточные условия существования и единственности решения задачи Коши~\eqref{eq:first_task}--\eqref{eq:start_condition} на промежутке $t_0 \leqslant t \leqslant t_1$ \cite{filippov}.

\begin{remark}
        Решением задачи Коши~\eqref{eq:first_task}--\eqref{eq:start_condition} мы называем \textit{решение Каратеодори}, то есть измеримую функцию $x(t)$, удовлетворяющую уравнениям~\eqref{eq:first_task}, \eqref{eq:start_condition}, а также интегральному уравнению
$$
        x(t) = x^0 + \int\limits_{t_0}^{t} [A(\tau)x(\tau) + B(\tau)u(\tau)]\,d\tau,
        \quad
        t_0\leqslant t \leqslant t_1,
$$ 
        где интеграл понимается в смысле Лебега.
\end{remark}