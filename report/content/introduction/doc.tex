\section{Введение}

Работа посвящена задаче построения линейно-квадратичного регулятора динамической системы в форме стратегии с запаздывающей обратной связью.

Данная задача прежде всего возникает при передаче сигнала на большие расстояния, либо при помощи компьютерных и прочих сетей.
В таких случаях информация о состоянии фазовых переменных поступает с некоторой задержкой.
В работе допускается, что эта задержка детерминированна и фазовые координаты передаются без помех.

Способом решения поставленной задачи предлагается метод динамического программирования, разработанный Р.~Беллманом~\cite{bellman}.
Данный метод предполагает синтезировать стратегию как минимизатор в уравнении Гамильтона--Якоби--Беллмана.

В работе представлен алгоритм построения оптимального управления в форме стратегии в случае непрерывной передачи данных, а также редукция к дискретной системе и соответствующий алгоритм в случае фиксированного интервала времени между отправками данных.
Представлены примеры работы программы для модели электродвигателя постоянного тока, и проведено сравнение с классическим решением для задачи без запаздыающей обратной связи.