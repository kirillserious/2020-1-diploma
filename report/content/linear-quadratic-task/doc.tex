\section{Непрерывная передача наблюдений}
\subsection{Формализация задачи}

Рассмотрим линейную динамическую систему с непрерывными ограниченными коэффициентами
\begin{equation}\label{eq:first_task}
        \frac{d}{dt}x(t) = A(t)x(t) + B(t)u(t),
        \quad
        t_0 \leqslant t \leqslant t_1
\end{equation}
с заданным начальным условием
\begin{equation}\label{eq:start_condition}
        x(t_0) = x^0.
\end{equation}
\begin{remark}
        Под решением задачи Коши~\eqref{eq:first_task}--\eqref{eq:start_condition} мы будем называть \textit{решение Каратеодори}, то есть измеримую функцию $x(t)$, удовлетворяющую также интегральному уравнению
$$
        x(t) = x^0 + \int\limits_{t_0}^{t} [A(\tau)x(\tau) + B(\tau)u(\tau)]\,d\tau,
        \quad
        t_0\leqslant t \leqslant t_1,
$$ 
        где интеграл понимается в смысле Лебега.
\end{remark}

Для задачи Коши~\eqref{eq:first_task}--\eqref{eq:start_condition} поставим задачу поиска измеримого управления, минимизируещего следующий интегрально-квадратичный функционал
\begin{equation}\label{eq:functional}
        J(u)
=
        \int\limits_{t_0}^{t_1}
[
\langle
x,\,M(s)x
\rangle
+
\langle
u,\,N(s)u
\rangle
]\,ds
        +
        \langle
        x(t_1),\,Tx(t_1)
        \rangle
\to
        \min\limits_{u \in U},
\end{equation}
где $M(s) = M\T(s) \geqslant 0$,
$N(s) = N\T(s) > 0$,
$T > 0$,
а $U$ --- множество измеримых функций на отрезке $t_0 \leqslant t \leqslant t_1$.

\begin{remark}
        Допустимый класс управления выбран таким образом, чтобы выполнялись достаточные условия существования и единственности решения задачи Коши~\eqref{eq:first_task}--\eqref{eq:start_condition} на промежутке $t_0 \leqslant t \leqslant t_1$ \cite{filippov}.
\end{remark}


\subsection{Решение задачи}
Обозначим за $\hat x(t,\,x^0,\,t_0)$ решение системы \eqref{eq:first_task} с заданным начальным условием $x(t_0) = x^0$ и введём в рассмотрение функцию цены
\begin{multline*}
        V(t,\,x)
=
\inf\limits_{u \in U} \left\{
        \int\limits_{t}^{t_1}
[
\langle
\hat x(s,\,t,\,x),\,M(s)\hat x(s,\,t,\,x)
\rangle
+
\langle
u(s),\,N(s)u(s)
\rangle
]\,ds
        \;+\right.\\+\left.
        \langle
        \hat x(t_1,\,t,\,x),\,T\hat x(t_1,\,t,\,x)
        \rangle
\right\}.
\end{multline*}
При этом $V(t_0,\,x^0) = \inf\limits_{u\in U} J(u)$. Тогда согласно \cite{krasovsky} такая функция цены удовлетворяет уравнению Гамильтона--Якоби--Беллмана:
\begin{equation}\label{eq:g-ya-b}
        \frac{\partial V}{\partial t}
        +
        \min\limits_{u}
        \left\{
        \underbrace{
\left\langle
\frac{\partial V}{\partial x}
,\,
Ax + Bu
\right\rangle
+
\langle
x,\,Mx
\rangle
+
\langle
u,\,Nu
\rangle
        }\limits_{\Psi}
        \right\}
        =
        0
\end{equation}
с краевым начальным условием
\begin{equation}\label{eq:g-ya-b-side}
        V(t_1,\,x)
        =
        \langle x,\,Tx \rangle.
\end{equation}
К тому же выполняется следующая теорема.
\begin{theorem}[О верификации]
        Пусть существует гладкая функция $W(t,\,x)$, удовлетворяющая уравнению Гамильтона--Якоби--Беллмана
$$
        \frac{\partial W}{\partial t}
        +\min\limits_{u}\Psi\left(\frac{\partial W}{\partial x},\,x,\,u\right) = 0
$$
с краевым начальным условием
$
        W(t_1,\,x) = \langle x,\,Tx \rangle,
$ причём минимум достигается на элементе $u^*$. Тогда $$V(t_0,\,x) = W(t_0,\,x).$$
\end{theorem}
Как следствие получаем, что измеримое управление $u^* = \arg\min\limits_u \Psi$, если такое существует, будет оптимальным для рассматриваемой задачи.

Так как получившаяся функция $\Psi$ строго выпукла по переменной $u$, то минимум будет достигнут в единственной точке. Запишем необходимое условие экстремума:
\begin{align*}
\mathrm{grad}_u \Psi\left(\frac{\partial V}{\partial x},\,x,\,u^*\right) &= 0,\\
B\T\frac{\partial V}{\partial x} + 2Nu^* &=0. 
\end{align*}
Получается, что оптимальное управление может быть задано как
\begin{equation}
        u^* = -\frac{1}{2}N^{-1}B\T\frac{\partial V}{\partial x}.
\end{equation}

Вернёмся к уравнению \eqref{eq:g-ya-b}, подставив в него получившийся результат:
$$
        \frac{\partial V}{\partial t}
        +
        \left\langle
        \frac{\partial V}{\partial x},\,Ax
        \right\rangle
        -
        \frac{1}{4}
        \left\langle
        \frac{\partial V}{\partial x},\,
        BN^{-1}B\T\frac{\partial V}{\partial x}
        \right\rangle
        +
        \langle
        x,\,Mx
        \rangle
        = 0.
$$
Будем искать функцию цены как квадратичную форму
$
        V(t,\,x)
        =
        \langle x,\,P(t)x\rangle,
$
где
$
        P(t) = P\T(t) > 0.
$
Тогда
\begin{align*}
\left\langle
x,\,\frac{dP}{dt} x
\right\rangle
+
2\langle
Px,\,Ax
\rangle
-
\langle
Px,\,BN^{-1}B\T P x
\rangle
+
\langle
x,\,Mx
\rangle
&= 0,
\\
\left\langle
x,\,\frac{dP}{dt} x
\right\rangle
+
\langle
x,\,PAx
\rangle
+
\langle
A\T P x,\,x
\rangle
-
\langle
Px,\,BN^{-1}B\T P x
\rangle
+
\langle
x,\,Mx
\rangle
&= 0,
\\
\left\langle
x,\,\frac{dP}{dt} x
\right\rangle
+
\langle
x,\,PAx
\rangle
+
\langle
x,\,A\T P x
\rangle
-
\langle
x,\,PBN^{-1}B\T P x
\rangle
+
\langle
x,\,Mx
\rangle
&= 0.
\end{align*}
Так как в правой части скалярных произведений стоят симметричные матрицы, мы можем записать следующее матричное дифференциальное уравнение, которому должна удовлетворять матрица $P$:
\begin{equation}\label{eq:rikkati}
\begin{cases}
\frac{dP}{dt} + PA + A\T P - PBN^{-1}B\T P + M = 0,\\
P(t_1) = T.
\end{cases}
\end{equation}
Уравнение \eqref{eq:rikkati} называют \textit{уравнением Риккати}. Подробно о свойствах решений уравнения Риккати можно прочитать в работе \cite{egorov}. Нам достаточно того, что для симметричных матриц решение данного уравнения существует и единственно в некоторой окрестности каждой точки $t$ рассматриваемого промежутка $t_0\leqslant t\leqslant t_1$.

Теперь вспомним, что информация о состоянии системы в текущий момент времени недоступна наблюдению. Воспользуемся формулой Коши:
$$
x(t) = X(t,\,t-h)\xi + \int\limits_{t-h}^{t}X(t,\,s)B(s)u(s)\,ds,
$$
где $\xi$ --- состояние системы в момент времени $t- h$, а $X(t, s)$ --- матрица Коши, определяющаяся следующими соотношениями:
$$
        \begin{cases}
\frac{\partial}{\partial t}X(t,\,s) = A(t)X(t,\,s),\\
X(s,\,s) = I.
        \end{cases}
$$

Таким образом, мы можем записать оптимальную стратегию
\begin{equation} \label{eq:opt_control}
u^*[t] = -N^{-1}(t)B\T(t) P(t)x[t],
\end{equation}
где матрица $P(t)$ является решением системы \eqref{eq:rikkati}, а 
\begin{equation*}
x[t] =
\begin{cases}
x_0,&\mbox{при }t = t_0\\
X(t,\,t_0)x_0 + \int\limits_{t_0}^{t}X(t,\,s)B(s)u^*[s]\,ds,
&\mbox{при }t_0 < t < t_0 + h,\\
X(t,\,t - h)x[t-h] + \int\limits_{t - h}^{t}X(t,\,s)B(s)u^*[s]\,ds,
&\mbox{при } t_0 + h < t \leqslant t_1.
\end{cases}
\end{equation*}
Отметим также, что с точки зрения поставленной задачи интегрирование управления до момента $t$ нам доступно, так как центр может хранить информацию о посланном ранее управлении.

Таким образом получили, что если отказаться от знания состояния в текущий момент времени, то для построения регулятора нам необходимо использовать как состояние в предыдущий момент времени $x(t - h)$, так и всё посланное управление $u(\tau)$ на интервале $t - h \leqslant \tau < t$.